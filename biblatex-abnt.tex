\documentclass[a4paper]{article}
\usepackage[brazil]{babel}
\usepackage{lmodern}
\usepackage[T1]{fontenc}
\usepackage[utf8]{inputenc}
\usepackage{microtype}
\usepackage{enumitem}
\usepackage{examplep}
\usepackage{hyperref}

\usepackage{parskip}
\setlength{\parskip}{1.2\baselineskip}

\usepackage{titlesec}
\newcommand{\subsectionbreak}{\clearpage}

\usepackage[
    backend=biber,
    style=abnt,
    %hyperref,         % Uncomment to turn citations into links
    %backref,          % Uncomment for back references (E.g. "Cit. on p. 1")
    %repeatfields,     % Uncomment to repeat fields instead of using an underscore
    %usedashes,        % Uncomment to use biblatex standard dashes instead of underscores
    giveninits,       % Uncomment to use initials for first names
    %bftitles,         % Uncomment to print titles in bold
    %indent,           % Uncomment to use hanging indentation in the bibliography
    %scbib,            % Uncomment to use small caps in the bibliography
    %sccite,           % Uncomment to use small caps in the citations
    %noslsn,           % Uncomment to hide [s.l], [s.n] and [s.l.: s.n.]
    %nosl,             % Uncomment to hide just [s.l]
    %nosn,             % Uncomment to hide just [s.n]
]{biblatex}
\usepackage[brazil]{babel}
\usepackage[autostyle]{csquotes}

\addbibresource{biblatex-abnt.bib}

\DeclareBibliographyCategory{singleentries}

\newcommand{\singlecite}[1]{%
  \addtocategory{singleentries}{#1}%
  \defbibcheck{key#1}{
    \iffieldequalstr{entrykey}{#1}
      {}
      {\skipentry}}%
  \printbibliography[heading=none,check=key#1]%
}

\title{biblatex-abnt}
\author{Daniel B. Marques}

\begin{document}

\maketitle

\tableofcontents

\clearpage
\section{Comandos}

\begin{description}[style=nextline]
    \item [\PVerb{\cite{bosi08}}] \cite{bosi08}
    \item [\PVerb{\textcite{bosi08}}] \textcite{bosi08}
    \item [\PVerb{\cite*{bosi08}}] \cite*{bosi08}
    \item [\PVerb{\textcite*{bosi08}}] \textcite*{bosi08}
    \item [\PVerb{\cites{moretti09}{mann09}{amaral15}}] \cites{moretti09}{mann09}{amaral15}
    \item [\PVerb{\textcites{moretti09}{mann09}{amaral15}}] \textcites{moretti09}{mann09}{amaral15}
    \item [\PVerb{\apud{assis08}{bosi08}}] \apud{assis08}{bosi08}
    \item [\PVerb{\apud[p.~12]{assis08}[p.~200]{bosi08}}] \apud[p.~12]{assis08}[p.~200]{bosi08}
    \item [\PVerb{\apud[batman][]{bosi08}}] \apud[batman][]{bosi08}
    \item [\PVerb{\textapud[p.~200]{assis08}[p.~12]{bosi08}}] \textapud[p.~200]{assis08}[p.~12]{bosi08}
\end{description}


\clearpage
\section{Opções}

As opções a seguir podem ser usadas ao chamar o pacote \texttt{biblatex}:

\begin{description}
    \item [hyperref] Transforma as citações em links que levam à bibliografia
    \item [backref] Aponta, na bibliografia, as páginas em que a entrada foi citada
    \item [repeatfields] Imprime os campos repetidos na bibliografia, em vez de substituí-los por traços sublineares
    \item [usedashes] Usa os traços padrão do \texttt{biblatex} em vez de traços sublineares nos campos repetidos
    \item [giveninits] Abrevia os primeiros nomes na bibliografia
    \item [bftitles] Usa negrito para os títulos na bibliografia
    \item [indent] Indenta as entradas da bibliografia
    \item [scbib] Imprime os nomes em versalete na bibliografia
    \item [sccite] Imprime os nomes em versalete nas citações
    \item [noslsn] Oculta as abreviações [s.l], [s.n] e [s.l.: s.n.] na bibliografia
    \item [nosl] Oculta apenas as abreviações [s.l.]
    \item [nosn] Oculta apenas as abreviações [s.n.]
\end{description}

E.g.: \verb"\usepackage[backend=biber, style=abnt, bftitles]{biblatex}"

As opções \texttt{repeatfields}, \texttt{nosl}, \texttt{nosn} e \texttt{noslsn} também podem ser usadas apenas em entradas específicas. E.g.:

\begin{verbatim}
    @mvbook{assis08,
        author = {Machado de Assis},
        title = {Obra completa em quatro volumes},
        year = {2008},
        options = {repeatfields, noslsn=false}
    }
\end{verbatim}


	
\clearpage
\section{Entradas comuns}

A lista completa de campos e entradas pode ser encontrada no manual do \texttt{biblatex}. Estes são alguns exemplos de situações comuns:

\begingroup
\let\clearpage\relax
\subsection{@mvbook}
\endgroup

Um livro abrangendo múltiplos volumes:
	
\begin{verbatim}
    @mvbook{assis08,
        author = {Machado de Assis},
        title = {Obra completa em quatro volumes},
        year = {2008},
        editor = {Aluizio Leite and Ana Lima Cecilio and Heloisa Jahn},
        editortype = {organizer},
        edition = {2},
        volumes = {4},
        publisher = {Nova Fronteira},
        location = {Rio de Janeiro},
        series = {Biblioteca luso-brasileira. Série brasileira},
    }
\end{verbatim}

\singlecite{assis08}
	
\subsection{@book}

Um único livro. Pode ser um dos volumes de um livro que abrange múltiplos volumes:
	
\begin{verbatim}
    @book{assis08:1,
        volume = {1},
        title = {Fortuna crítica/Romance},
        pagetotal = {1340},
        author = {Machado de Assis},
        maintitle = {Obra completa em quatro volumes},
        year = {2008},
        editor = {Aluizio Leite and Ana Lima Cecilio and Heloisa Jahn},
        editortype = {organizer},
        edition = {2},
        publisher = {Nova Fronteira},
        location = {Rio de Janeiro},
        series = {Biblioteca luso-brasileira. Série brasileira},
    }
\end{verbatim}

\noindent
Também é possível usar o campo \texttt{crossref} para herdar as informações de outra entrada:
	
\begin{verbatim}
    @book{assis08:1,
        crossref = {assis08},
        volume = {1},
        title = {Fortuna crítica/Romance},
        pagetotal = {1340},
    }
\end{verbatim}

\singlecite{assis08:1}
	
\subsection{@bookinbook}
	
Uma obra originalmente publicada por si só, mas citada como parte de outro livro:

\begin{verbatim}
    @bookinbook{assis08:1b,
        title = {Esaú e Jacó},
        pages = {1073-1226},
        volume = {1},
        booktitle = {Fortuna crítica/Romance},
        pagetotal = {1340},
        author = {Machado de Assis},
        maintitle = {Obra completa em quatro volumes},
        year = {2008},
        editor = {Aluizio Leite and Ana Lima Cecilio and Heloisa Jahn},
        editortype = {organizer},
        edition = {2},
        publisher = {Nova Fronteira},
        location = {Rio de Janeiro},
        series = {Biblioteca luso-brasileira. Série brasileira},
    }
\end{verbatim}

\noindent
Ou:
	
\begin{verbatim}
    @bookinbook{assis08:1b,
        crossref = {assis08:1},
        title = {Esaú e Jacó},
        pages = {1073-1226},
    }
\end{verbatim}

\singlecite{assis08:1b}
	
\subsection{@inbook}

Uma parte de um livro que forma uma unidade independente, com seu próprio título:
	
\begin{verbatim}
    @inbook{bosi08,
        title = {Uma figura machadiana},
        author = {Alfredo Bosi},
        pages = {179-189},
        volume = {1},
        booktitle = {Fortuna crítica/Romance},
        pagetotal = {1340},
        bookauthor = {Machado de Assis},
        maintitle = {Obra completa em quatro volumes},
        year = {2008},
        editor = {Aluizio Leite and Ana Lima Cecilio and Heloisa Jahn},
        editortype = {organizer},
        edition = {2},
        publisher = {Nova Fronteira},
        location = {Rio de Janeiro},
        series = {Biblioteca luso-brasileira. Série brasileira},
    }
\end{verbatim}

\noindent
Ou:
	
\begin{verbatim}
    @inbook{bosi08,
        crossref = {assis08:1},
        title = {Uma figura machadiana},
        author = {Alfredo Bosi},
        pages = {179-189},
    }
\end{verbatim}

\singlecite{bosi08}
	
\subsection{@suppbook}

Uma parte suplementar de um livro, com um título genérico, como ``prefácio'' ou ``introdução'':
	
\begin{verbatim}
    @suppbook{leite08,
        title = {Nota Editorial},
        author = {Aluizio Leite and Ana Lima Cecilio and Heloisa Jahn},
        pages = {1-5},
        volume = {1},
        booktitle = {Fortuna crítica/Romance},
        pagetotal = {1340},
        bookauthor = {Machado de Assis},
        maintitle = {Obra completa em quatro volumes},
        year = {2008},
        editor = {Aluizio Leite and Ana Lima Cecilio and Heloisa Jahn},
        editortype = {organizer},
        edition = {2},
        publisher = {Nova Fronteira},
        location = {Rio de Janeiro},
        series = {Biblioteca luso-brasileira. Série brasileira},
    }
\end{verbatim}

\noindent
Ou:
	
\begin{verbatim}
    @suppbook{leite08,
        crossref = {assis08:1},
        title = {Nota Editorial},
        author = {Aluizio Leite and Ana Lima Cecilio and Heloisa Jahn},
        pages = {1-5},
    }
\end{verbatim}

\singlecite{leite08}
	
\subsection{@mvcollection}

Uma coleção abrangendo diversos volumes, cada um composto por diversas contribuições independentes, com seus próprios autor e títulos. A obra como um todo não possui um autor, mas geralmente possui um editor:
	
\begin{verbatim}
    @mvcollection{moretti09,
        editor = {Franco Moretti},
        editortype = {organizer},
        translator = {Denise Bottmann},
        title = {O Romance},
        volumes = {5},
        publisher = {Cosac Naify},
        location = {São Paulo},
        year = {2009},
    }
\end{verbatim}

\singlecite{moretti09}
	
\subsection{@collection}

Uma única coleção composta por diversas contribuições independentes. Pode ser um dos volumes de uma coleção que abrange múltiplos volumes:
	
\begin{verbatim}
    @collection{moretti09:1,
        volume = {1},
        title = {A cultura do romance},
        pagetotal = {1120},
        illustrated = {40 ils.},
        editor = {Franco Moretti},
        editortype = {organizer},
        translator = {Denise Bottmann},
        maintitle = {O Romance},
        publisher = {Cosac Naify},
        location = {São Paulo},
        year = {2009},
    }
\end{verbatim}

\noindent
Ou:
	
\begin{verbatim}
    @collection{moretti09:1,
        crossref = {moretti09},
        volume = {1},
        title = {A cultura do romance},
        pagetotal = {1120},
        illustrated = {40 ils.},
    }
\end{verbatim}

\singlecite{moretti09:1}
	
\subsection{@incollection}

Uma contribuição a uma coleção, formando uma unidade independente com autor e título próprios:
	
\begin{verbatim}
    @incollection{mann09,
        author = {Thomas Mann},
        title = {Bilse e eu},
        pages = {217},
        volume = {1},
        booktitle = {A cultura do romance},
        pagetotal = {1120},
        illustrated = {40 ils.},
        editor = {Franco Moretti},
        editortype = {organizer},
        translator = {Denise Bottmann},
        maintitle = {O Romance},
        publisher = {Cosac Naify},
        location = {São Paulo},
        year = {2009},
    }
\end{verbatim}

\noindent
Ou:
	
\begin{verbatim}
    @incollection{mann09,
        crossref = {moretti09:1},
        author = {Thomas Mann},
        title = {Bilse e eu},
        pages = {217},
    }
\end{verbatim}

\singlecite{mann09}
	
\subsection{@suppcollection}

Uma parte suplementar de uma coleção, com um título genérico, como ``prefácio'' ou ``introdução'':
	
\begin{verbatim}
    @suppcollection{moretti09:1b,
        title = {Apresentação geral},
        author = {Franco Moretti},
        pages = {217},
        volume = {1},
        booktitle = {A cultura do romance},
        pagetotal = {1120},
        illustrated = {40 ils.},
        editor = {Franco Moretti},
        editortype = {organizer},
        translator = {Denise Bottmann},
        maintitle = {O Romance},
        publisher = {Cosac Naify},
        location = {São Paulo},
        year = {2009},
    }
\end{verbatim}

\noindent
Ou:
	
\begin{verbatim}
    @suppcollection{moretti09:1b,
        crossref = {moretti09:1},
        title = {Apresentação geral},
        author = {Franco Moretti},
    }
\end{verbatim}

\singlecite{moretti09:1b}

\subsection{@article}

Um artigo científico/acadêmico: 
	
\begin{verbatim}
    @article{negrão14,
        title = {Brazilian Portuguese as a transatlantic language},
        subtitle = {agents of linguistic contact},
        author = {Esmeralda Vailati Negrão and Evani Viotti},
        journal = {Interdisciplinary Journal of Portuguese Diaspora Studies},
        volume = {3},
        pages = {135-154},
        year = {2014},
    }
\end{verbatim}

\singlecite{negrão14}

\subsection{@mastersthesis}

Uma dissertação de mestrado:
	
\begin{verbatim}
    @mastersthesis{eliseu84,
        title = {Verbos ergativos do Português},
        subtitle = {descrição e análise},
        author = {André Manuel Godinho Simões Eliseu},
        type = {Dissertação (Mestrado em Linguística)},
        institution = {Universidade de Lisboa},
        location = {Lisboa},
        eventyear = {1985},
    }
\end{verbatim}

\singlecite{eliseu84}

\subsection{@phdthesis}

Uma tese de doutorado:
	
\begin{verbatim}
    @phdthesis{amaral15,
        title = {A alternância transitivo-intransitiva no português brasileiro},
        subtitle = {fenômenos semânticos},
        author = {Luana Lopes Amaral},
        type = {Tese (Doutorado em Linguística)},
        institution = {Universidade Federal de Minas Gerais},
        location = {Belo Horizonte},
        eventyear = {2015},
    }
\end{verbatim}

\singlecite{amaral15}

\subsection{@inproceedings}

Resumos ou anais de eventos:
	
\begin{verbatim}
    @inproceedings{negrão13,
        title = {A emergência da sintaxe do português brasileiro},
        subtitle = {absolutas, alçamento do possuidor e passivas},
        author = {Esmeralda Vailati Negrão and Evani Viotti},
        eventtitle = {Encontro nacional do gt de teoria da gramática da ANPOLL},
        number = {28},
        venue = {Florianópolis},
        eventyear = {2013},
        booktitle = {Caderno de Resumos},
        publisher = {},
        location = {Campinas},
        year = {2013},
    }
\end{verbatim}

\singlecite{negrão13}

\clearpage
\nocite{*}
\printbibliography

\end{document}
