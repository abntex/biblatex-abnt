%% Copyright 2016 Daniel Ballester Marques
%%
%% This work may be distributed and/or modified under the
%% conditions of the LaTeX Project Public License, either version 1.3
%% of this license or (at your option) any later version.
%% The latest version of this license is in
%%   http://www.latex-project.org/lppl.txt
%% and version 1.3 or later is part of all distributions of LaTeX
%% version 2005/12/01 or later.
%%
%% This work has the LPPL maintenance status `maintained'.
%%
%% The Current Maintainer of this work is Daniel Ballester Marques.


% Preamble >>>1

\documentclass[a4paper]{article}
\usepackage[brazil]{babel}
\usepackage{lmodern}
\usepackage{microtype}
\usepackage{enumitem}
\usepackage{examplep}
\usepackage[T1]{fontenc}
\usepackage[utf8]{inputenc}
\usepackage[autostyle]{csquotes}

\usepackage{parskip}% >>>2
\setlength{\parskip}{\baselineskip}% <<<2

\usepackage{titlesec}% >>>2
\newcommand{\subsectionbreak}{\clearpage}% <<<2

\usepackage{tcolorbox}% >>>2
\usepackage{xcolor}
\tcbuselibrary{listings}

\newtcblisting{example}{
    colback=white,
    colframe=white,
    size=title,
    middle=-1pt,
    top=-2pt,
    left=0pt,
    fontlower=\small,
    left skip=2em,
}% <<<2


\setlength\itemsep{.6\baselineskip}

\usepackage[%% Biblatex >>>2
    style = abnt,
    language = brazil,
    giveninits,
    uniquename = init,
    hyperref,
    labelnumber,            % These two options are used just to show the
    defernumbers = true,    % numbers properly in the abnt-numeric examples
]{biblatex}

\addbibresource{biblatex-abnt.bib}% <<<2

\usepackage{hyperref}% >>>2
% \hypersetup{
%   colorlinks=true,
%   linkcolor=blue,
%   citecolor=red,
% } <<<2


\DeclareBibliographyCategory{singleentries}% >>>2

\newcommand{\singlecite}[1]{%
  \addtocategory{singleentries}{#1}%
  \defbibcheck{key#1}{
    \iffieldequalstr{entrykey}{#1}
      {}
      {\skipentry}}%
  \printbibliography[heading=none,check=key#1]%
}% <<<2

\newcommand{\version}{3.4}
\newcommand{\requirements}{\texttt{biblatex 3.8} e \texttt{biber 2.8}}

\title{biblatex-abnt \version}
\author{Daniel Ballester Marques}

% <<<1

\begin{document}% >>>1

\maketitle

\tableofcontents

\clearpage
\section{Requisitos}% >>>2

O \texttt{biblatex-abnt \version} requer \requirements. Caso haja algum
problema na compilação, cheque se seus pacotes estão atualizados.
% <<<2

\section{Instalação}% >>>2

O \texttt{biblatex-abnt} está incluso no TeX Live a partir 2016.

Para instalá-lo manualmente, copie os arquivos \texttt{.bbx}, \texttt{.cbx} e
\texttt{.lbx} para
\begin{verbatim}
    <TEXMFLOCAL>/tex/latex/biblatex-contrib/biblatex-abnt/
\end{verbatim}
e atualize o banco de dados do TeX (rodando o `texhash`, por exemplo).
% <<<2

\section{Uso}% >>>2

Para usar o {biblatex-abnt}, adicione as seguintes linhas ao preâmbulo do seu
arquivo {.tex}:

\begin{verbatim}
    \usepackage[
        style    = abnt,           % Sistema alfabético
      % style    = abnt-numeric,   % Sistema numérico
      % style    = abnt-ibid,      % Notas de referência
    ]{biblatex}

    \addbibresource{arquivo.bib}        % Seus arquivos de
    \addbibresource{outroarquivo.bib}   % bibliografia vão aqui.
\end{verbatim}

Após a opção \texttt{style}, podem ser acrescentadas
as opções descritas na seção \ref{sec:opções}.

Use os comandos descritos na seção \ref{sec:comandos} para citar obras.

Use o comando \verb"\printbibliography" para imprimir a bibliografia.
% <<<2

\clearpage
\section{Opções}% >>>2
\label{sec:opções}

As opções a seguir podem ser usadas ao chamar o \texttt{biblatex} (além
daquelas descritas no manual do pacote):

\begin{description}[itemindent=-1em,leftmargin=3em]% >>>3
  \item [ittitles] Usa itálico para os títulos na bibliografia
  \item [justify] Imprime o texto justificado em vez de alinhado à esquerda
  \item [indent] Indenta as entradas da bibliografia
  \item [sccite] Imprime os nomes em versalete nas citações
  \item [scbib] Imprime os nomes em versalete na bibliografia
  \item [pretty] Ativa as cinco primeiras opções citadas de uma vez
  \item [oldspacing] Não realiza as alterações de espaçamento entre
    referências bibliográficas introduzidas na norma NBR 6023 2018
    (espaçamento simples entre linhas e uma linha em branco entre
    entradas)
  \item [giveninits] Abrevia os primeiros nomes na bibliografia
  \item [extrayear] Diferencia os anos com letras (e.g., 2017a)
    também na bibliografia
  \item [nosl] Oculta as abreviações [s.l.] na bibliografia
  \item [nosn] Oculta as abreviações [s.n.] na bibliografia
  \item [noslsn] Oculta as abreviações [s.l], [s.n] e [s.l.: s.n.]
  \item [repeatfields] Imprime os campos repetidos na bibliografia, em vez de
    subs\-ti\-tu\-í-los por traços sublineares
  \item [repeatfirstfields] Repete apenas o primeiro campo de cada entrada,
    mas substitui uma segunda ocorrência com traços sublineares
  \item [repeattitles] Imprime apenas os títulos repetidos na bibliografia,
    mas substitui os autores por traços sublineares
  \item [usedashes] Usa os traços padrão do \texttt{biblatex} nos campos
    repetidos
  \item [slashdaterange] Utiliza uma barra ao invés do caractere padrão para separar períodos de datas. Sua utilização é mais recomendada para apenas entradas específicas
  \item [language=brazil] Essa opção é adicionada automaticamente. Para
    imprimir a bibliografia em outros idiomas, substitua o termo
    \texttt{brazil} pelo código da linguagem desejada
  \item [backref] Especifica as páginas em que cada entrada foi citada.
  \item [citecount] Além das páginas, especifica quantas vezes cada entrada
    foi citada.
  \item [comp] Abrevia a numeração no estilo numérico, imprimindo, e.g.,
    \enquote{(1--4)} em vez de \enquote{(1, 2, 3, 4)}. Para funcionar, todas as entradas devem ser adicionadas dentro do mesmo par de chaves, sendo separadas por vírgula, e.g.: \verb|\cite{lucca2009,assis08,assis08:1,assis08:1b}|.
\end{description}% <<<3

\vspace{\baselineskip}
E.g.: \verb"\usepackage[backend=biber, style=abnt, ittitles]{biblatex}"

\begin{sloppypar}
  As opções \texttt{repeatfields}, \texttt{repeattitles}, \texttt{backref},
  \texttt{nosl}, \texttt{nosn}, \texttt{noslsn}, \texttt{extrayear} e \texttt{slashdaterange} também
  podem ser usadas apenas em entradas específicas. E.g.:
\end{sloppypar}

\begin{verbatim}
    @mvbook{assis08,
        author  = {Machado de Assis},
        title   = {Obra completa em quatro volumes},
        year    = {2008},
        options = {repeattitles, noslsn=false},
    }
\end{verbatim}

Ou, um exemplo de uso do \texttt{slashdaterange}:
\begin{verbatim}
    @article{lucca2009,
        author            = {Gabriella {De Lucca}},
        title             = {Notas curtas},
        journaltitle      = {Getulio},
        location          = {São Paulo},
        volume            = {ano 3},
        pages             = {9},
        date              = {2009-07/2009-08},  
        options           = {slashdaterange}
    }
\end{verbatim}

    Que resulta em:
    \singlecite{lucca2009}
    
% <<<2


\section{Comandos}% >>>2
\label{sec:comandos}

\subsection{Estilo \texttt{abnt}}% >>>3

Comandos principais:

\begin{example}
\cite{amaral15}
\end{example}

\begin{example}
\textcite{bosi08}
\end{example}

\begin{example}
\apud{assis08}{bosi08}
\end{example}

\begin{example}
\cites{moretti09}{mann09}{amaral15}
\end{example}

\begin{example}
\textcites{moretti09}{mann09}{amaral15}
\end{example}

Outros exemplos:

\begin{example}
Como sabemos\footcite[Cf., e.g.,][]{assis08},
\end{example}

\begin{example}
\citetitle{bosi08}
\end{example}

\begin{example}
\citeauthor{bosi08}
\end{example}

\begin{example}
\citeyear{bosi08}
\end{example}

\begin{example}
\cites{mann09}{moretti09:1, moretti09}
\end{example}

\begin{example}
\apud[p.~12]{assis08}[p.~200]{bosi08}
\end{example}

\begin{example}
\textapud[p.~200]{assis08}[p.~12]{bosi08}
\end{example}

\begin{example}
\apud[batman][]{bosi08}
\end{example}

\begin{example}
Assis \cite[apud][p.~200]{bosi08}
\end{example}

% <<<3


\begingroup
\let\clearpage\relax
\subsection{Estilo \texttt{abnt-numeric}}% >>>3
\endgroup

\DeclareCiteCommand{\supercite}[\mkbibsuperscript]
  {\iffieldundef{prenote}
     {}
     {\BibliographyWarning{Ignoring prenote argument}}%
   \iffieldundef{postnote}
     {}
     {\BibliographyWarning{Ignoring postnote argument}}}
  {\usebibmacro{citeindex}%
   \usebibmacro{cite}}
  {\supercitedelim}
  {}

\providebool{bbx:subentry}

\renewbibmacro*{cite}{%
  \printtext[bibhyperref]{%
    \printfield{labelprefix}%
    \printfield{labelnumber}%
    \ifbool{bbx:subentry}
      {\printfield{entrysetcount}}
      {}}}

\makeatletter
\renewbibmacro*{textcite}{%
  \let\mkbibnamefamily\origmkbibnamefamily%
  \let\mkbibnamegiven\origmkbibnamegiven%
  \let\mkbibnameprefix\origmkbibnameprefix%
  \let\mkbibnamesuffix\origmkbibnamesuffix%
  \iffieldequals{namehash}{\cbx@lasthash}
    {\setunit{\multicitedelim}}
    {\ifnameundef{labelname}
       {\printfield[citetitle]{labeltitle}}
       {\printnames{labelname}}%
     \setunit{%
       \global\booltrue{cbx:parens}%
       \printdelim{namelabeldelim}\bibopenparen}%
     \stepcounter{textcitecount}%
     \savefield{namehash}{\cbx@lasthash}}%
  \ifnumequal{\value{citecount}}{1}
    {\usebibmacro{prenote}}
    {}%
  \usebibmacro{cite}%
  \setunit{%
    \ifbool{cbx:parens}
      {\bibcloseparen\global\boolfalse{cbx:parens}}
      {}%
    \textcitedelim}}
\makeatother


\begin{example}
\textcite{bosi08}
\end{example}

\begin{example}
Bosi \cite{bosi08}
\end{example}

\begin{example}
Bosi \supercite{bosi08}
\end{example}
% <<<3


\begingroup
\let\clearpage\relax
\subsection{Estilo \texttt{abnt-ibid}}% >>>3
\endgroup

No estilo \texttt{abnt-ibid}, o comando \texttt{footcite} imprime, conforme as
instruções da ABNT, a referência inteira, sua versão abreviada ou as
abreviações ``id.'', ``ibid.'', ``op. cit.'' e ``loc. cit.''. O comando
\texttt{apud} também é adaptado para usar as notas de rodapé.

% <<<3

% <<<2

\clearpage
\section{Entradas comuns}% >>>2

A lista completa de campos e entradas pode ser encontrada no manual do
\texttt{biblatex}. Estes são alguns exemplos de situações comuns:

\begingroup
\let\clearpage\relax
\subsection{@mvbook}% >>>3
\endgroup

Um obra que abrange múltiplos volumes:

\begin{verbatim}
    @mvbook{assis08,
        author     = {Machado de Assis},
        title      = {Obra completa em quatro volumes},
        year       = {2008},
        editor     = {Aluizio Leite and
                      Ana Lima Cecilio and Heloisa Jahn},
        editortype = {organizer},
        edition    = {2},
        volumes    = {4},
        publisher  = {Nova Fronteira},
        location   = {Rio de Janeiro},
        series     = {Biblioteca luso-brasileira.
                      Série brasileira},
    }
\end{verbatim}

\singlecite{assis08}
% <<<3

\subsection{@book}% >>>3

Um único livro. Pode ser um dos volumes de uma obra que abrange múltiplos
volumes (\texttt{@mvbook}):

\begin{verbatim}
    @book{assis08:1,
        volume     = {1},
        title      = {Fortuna crítica/Romance},
        pagetotal  = {1340},
        author     = {Machado de Assis},
        maintitle  = {Obra completa em quatro volumes},
        year       = {2008},
        editor     = {Aluizio Leite and
                      Ana Lima Cecilio and Heloisa Jahn},
        editortype = {organizer},
        edition    = {2},
        publisher  = {Nova Fronteira},
        location   = {Rio de Janeiro},
        series     = {Biblioteca luso-brasileira.
                      Série brasileira},
    }
\end{verbatim}

\noindent
Também é possível usar o campo \texttt{crossref} para herdar as informações de
outra entrada:

\begin{verbatim}
    @book{assis08:1,
        crossref  = {assis08},
        volume    = {1},
        title     = {Fortuna crítica/Romance},
        pagetotal = {1340},
    }
\end{verbatim}

\singlecite{assis08:1}
% <<<3

\subsection{@bookinbook}% >>>3

Uma obra originalmente publicada por si só, mas citada como parte de outro
livro:

\begin{verbatim}
    @bookinbook{assis08:1b,
        title      = {Esaú e Jacó},
        pages      = {1073-1226},
        volume     = {1},
        booktitle  = {Fortuna crítica/Romance},
        pagetotal  = {1340},
        author     = {Machado de Assis},
        maintitle  = {Obra completa em quatro volumes},
        year       = {2008},
        editor     = {Aluizio Leite and
                      Ana Lima Cecilio and Heloisa Jahn},
        editortype = {organizer},
        edition    = {2},
        publisher  = {Nova Fronteira},
        location   = {Rio de Janeiro},
        series     = {Biblioteca luso-brasileira.
                      Série brasileira},
    }
\end{verbatim}

\noindent
Ou:

\begin{verbatim}
    @bookinbook{assis08:1b,
        crossref = {assis08:1},
        title    = {Esaú e Jacó},
        pages    = {1073-1226},
    }
\end{verbatim}

\singlecite{assis08:1b}
% <<<3

\subsection{@inbook}% >>>3

Uma parte de um livro que forma uma unidade independente, com seu próprio
título:

\begin{verbatim}
    @inbook{bosi08,
        title      = {Uma figura machadiana},
        author     = {Alfredo Bosi},
        pages      = {179-189},
        volume     = {1},
        booktitle  = {Fortuna crítica/Romance},
        pagetotal  = {1340},
        bookauthor = {Machado de Assis},
        maintitle  = {Obra completa em quatro volumes},
        year       = {2008},
        editor     = {Aluizio Leite and
                      Ana Lima Cecilio and Heloisa Jahn},
        editortype = {organizer},
        edition    = {2},
        publisher  = {Nova Fronteira},
        location   = {Rio de Janeiro},
        series     = {Biblioteca luso-brasileira.
                      Série brasileira},
    }
\end{verbatim}

\noindent
Ou:

\begin{verbatim}
    @inbook{bosi08,
        crossref = {assis08:1},
        title    = {Uma figura machadiana},
        author   = {Alfredo Bosi},
        pages    = {179-189},
    }
\end{verbatim}

\singlecite{bosi08}
% <<<3

\subsection{@suppbook}% >>>3

Uma parte suplementar de um livro, com um título genérico, como ``prefácio''
ou ``introdução'':

\begin{verbatim}
    @suppbook{leite08,
        title      = {Nota Editorial},
        author     = {Aluizio Leite and
                      Ana Lima Cecilio and Heloisa Jahn},
        pages      = {1-5},
        volume     = {1},
        booktitle  = {Fortuna crítica/Romance},
        pagetotal  = {1340},
        bookauthor = {Machado de Assis},
        maintitle  = {Obra completa em quatro volumes},
        year       = {2008},
        editor     = {Aluizio Leite and
                      Ana Lima Cecilio and Heloisa Jahn},
        editortype = {organizer},
        edition    = {2},
        publisher  = {Nova Fronteira},
        location   = {Rio de Janeiro},
        series     = {Biblioteca luso-brasileira.
                      Série brasileira},
    }
\end{verbatim}

\noindent
Ou:

\begin{verbatim}
    @suppbook{leite08,
        crossref = {assis08:1},
        title    = {Nota Editorial},
        author   = {Aluizio Leite and
                    Ana Lima Cecilio and Heloisa Jahn},
        pages    = {1-5},
    }
\end{verbatim}

\singlecite{leite08}
% <<<3

\subsection{@mvcollection}% >>>3

Uma coleção abrangendo diversos volumes, cada um composto por diversas
contribuições independentes, com seus próprios autores e títulos. A obra como
um todo não possui um autor, mas geralmente possui um editor:

\begin{verbatim}
    @mvcollection{moretti09,
        editor     = {Franco Moretti},
        editortype = {organizer},
        translator = {Denise Bottmann},
        title      = {O Romance},
        volumes    = {5},
        publisher  = {Cosac Naify},
        location   = {São Paulo},
        year       = {2009},
    }
\end{verbatim}

\singlecite{moretti09}
% <<<3

\subsection{@collection}% >>>3

Um único livro constituindo uma coleção composta por diversas contribuições
independentes. Pode ser um dos volumes de uma coleção que abrange múltiplos
volumes (\texttt{@mvcollection}):

\begin{verbatim}
    @collection{moretti09:1,
        volume      = {1},
        title       = {A cultura do romance},
        pagetotal   = {1120},
        illustrated = {40 ils.},
        editor      = {Franco Moretti},
        editortype  = {organizer},
        translator  = {Denise Bottmann},
        maintitle   = {O Romance},
        publisher   = {Cosac Naify},
        location    = {São Paulo},
        year        = {2009},
    }
\end{verbatim}

\noindent
Ou:

\begin{verbatim}
    @collection{moretti09:1,
        crossref    = {moretti09},
        volume      = {1},
        title       = {A cultura do romance},
        pagetotal   = {1120},
        illustrated = {40 ils.},
    }
\end{verbatim}

\singlecite{moretti09:1}
% <<<3

\subsection{@incollection}% >>>3

Uma contribuição a uma coleção, formando uma unidade independente, com autor
e título próprios:

\begin{verbatim}
    @incollection{mann09,
        author      = {Thomas Mann},
        title       = {Bilse e eu},
        pages       = {217},
        volume      = {1},
        booktitle   = {A cultura do romance},
        pagetotal   = {1120},
        illustrated = {40 ils.},
        editor      = {Franco Moretti},
        editortype  = {organizer},
        translator  = {Denise Bottmann},
        maintitle   = {O Romance},
        publisher   = {Cosac Naify},
        location    = {São Paulo},
        year        = {2009},
    }
\end{verbatim}

\noindent
Ou:

\begin{verbatim}
    @incollection{mann09,
        crossref = {moretti09:1},
        author   = {Thomas Mann},
        title    = {Bilse e eu},
        pages    = {217},
    }
\end{verbatim}

\singlecite{mann09}
% <<<3

\subsection{@suppcollection}% >>>3

Uma parte suplementar de uma coleção, com um título genérico, como
``prefácio'' ou ``introdução'':

\begin{verbatim}
    @suppcollection{moretti09:1b,
        title       = {Apresentação geral},
        author      = {Franco Moretti},
        pages       = {217},
        volume      = {1},
        booktitle   = {A cultura do romance},
        pagetotal   = {1120},
        illustrated = {40 ils.},
        editor      = {Franco Moretti},
        editortype  = {organizer},
        translator  = {Denise Bottmann},
        maintitle   = {O Romance},
        publisher   = {Cosac Naify},
        location    = {São Paulo},
        year        = {2009},
    }
\end{verbatim}

\noindent
Ou:

\begin{verbatim}
    @suppcollection{moretti09:1b,
        crossref = {moretti09:1},
        title    = {Apresentação geral},
        author   = {Franco Moretti},
    }
\end{verbatim}

\singlecite{moretti09:1b}
% <<<3

\subsection{@article}% >>>3

Um artigo científico/acadêmico:

\begin{verbatim}
    @article{negrão14,
        title    = {Brazilian Portuguese as a
                    transatlantic language},
        subtitle = {agents of linguistic contact},
        author   = {Esmeralda Vailati Negrão and Evani Viotti},
        journal  = {Interdisciplinary Journal of
                    Portuguese Diaspora Studies},
        volume   = {3},
        pages    = {135-154},
        year     = {2014},
    }
\end{verbatim}

\singlecite{negrão14}
% <<<3

\subsection{@thesis}% >>>3

Uma dissertação de mestrado:

\begin{verbatim}
    @thesis{eliseu84,
        title       = {Verbos ergativos do Português},
        subtitle    = {descrição e análise},
        author      = {André Manuel Godinho Simões Eliseu},
        type        = {Dissertação (Mestrado em Linguística)},
        institution = {Universidade de Lisboa},
        location    = {Lisboa},
        eventyear   = {1985},
    }
\end{verbatim}

\singlecite{eliseu84}

Uma tese de doutorado:

\begin{verbatim}
    @thesis{amaral15,
        title       = {A alternância transitivo-intransitiva
                       no português brasileiro},
        subtitle    = {fenômenos semânticos},
        author      = {Luana Lopes Amaral},
        type        = {Tese (Doutorado em Linguística)},
        institution = {Universidade Federal de Minas Gerais},
        location    = {Belo Horizonte},
        eventyear   = {2015},
    }
\end{verbatim}

\singlecite{amaral15}
% <<<3

\subsection{@inproceedings}% >>>3

Trabalhos publicados em resumos ou anais de eventos:

\begin{verbatim}
    @inproceedings{negrão13,
        title      = {A emergência da sintaxe do
                      português brasileiro},
        subtitle   = {absolutas, alçamento do
                      possuidor e passivas},
        author     = {Esmeralda Vailati Negrão and Evani Viotti},
        eventtitle = {Encontro nacional do gt de
                      teoria da gramática da ANPOLL},
        number     = {28},
        venue      = {Florianópolis},
        eventyear  = {2013},
        booktitle  = {Caderno de Resumos},
        publisher  = {ANPOLL},
        location   = {Campinas},
        year       = {2013},
    }
\end{verbatim}

\singlecite{negrão13}
% <<<3

% <<<2

% \clearpage
% \section{Compatibilidade com o pacote \texttt{abntex2cite}}% >>>2

% Em geral é possível usar o mesmo arquivo \texttt{.bib} utilizado pelo
% \texttt{abntex2cite}, e para a maior parte das entradas mais simples nenhuma
% mudança é necessária; estas são as principais exceções:

% \begin{itemize}
%   \begin{sloppypar}
%   \item Não consegui entender por que, mas os exemplos do
%     \texttt{abntex2cite} usam \verb"\'\i" para o caractere ``í'', enquanto o
%     normal me parece ser \verb"\'i" (para outras letras com acento agudo o
%     \texttt{abntex2cite} usa o formato normal). Isso pode causar alguns
%     problemas, o ideal é usar \verb"\'i" (ou aproveitar os recursos do biber
%     e usar a codificação utf-8). Caso isso não seja possível por algum
%     motivo, chamar o biblatex com a opção \texttt{safeinputenc} pode
%     resolver alguns problemas (mas também causa alguns outros, interferindo,
%     por exemplo, na capitalização automática de palavras acentuadas).
%   \item Os campos de datas (\texttt{date}, \texttt{year}, \texttt{month} etc.)
%     devem seguir o formato usado pelo \texttt{biblatex} (\texttt{yyyy-mm-dd}).
%     Cf. seção 2.2.1 do manual.
%   \item O \texttt{biblatex} diferencia os campos \texttt{pages} (e.g. ``p.
%     12-18'') e \texttt{pagetotal} (e.g. ``347 p.'').
%   \item Quando o primeiro campo impresso é o título, a primeira palavra é
%     automaticamente impressa em maiúsculas. Para que mais de uma palavra seja
%     impressa em maiúsculas, coloque-as entre chaves ou separadas por um
%     \emph{no breaking space} (\verb"~").
%   \item Quando o primeiro campo impresso é a organização, todo o campo é
%     impresso em maiúsculas. Para que a capitalização de uma palavra não
%     seja alterada, coloque-a entre chaves (e.g. \verb"{pAlaVRa}").
%   \item Pode-se incluir no campo \texttt{options} as opções \texttt{nosl}
%     (para não mostrar a abreviação ``[s.l]''), \texttt{nosn} (para não
%     mostrar ``[s.n.]'') ou \texttt{noslsn} (para não mostrar nenhuma das duas
%     abreviações) (e.g. \texttt{options = \{noslsn\}}); essa configuração
%     tem efeito em cada entrada específica. Pode-se também incluir essas
%     opções
%     ao chamar o \texttt{biblatex} para nunca mostrar as abreviações, e
%     então abrir exceções para entradas específicas (e.g. \texttt{options =
%     \{noslsn=false\}}).
%   \item Para periódicos, deve se utilizar entradas \texttt{@periodical}.
%     Usando-se \texttt{@book}, como no \texttt{abntex2cite}, o ISSN não
%     aparece.
%   \item Para teses de mestrado e doutorado, o campo \texttt{pagetotal} é
%     automaticamente formatado em folhas, como requer a norma (e.g. ``347
%     f.''). Para usar páginas também nessas entradas, use
%     \texttt{bookpagination =
%     \{page\}}. Para usar folhas em outras entradas, use
%     \texttt{bookpagination = \{sheet\}}.
%   \item Ao modificar o campo \texttt{type} em teses de mestrado e doutorado,
%     deve-se preenchê-lo com todo o texto desejado (e.g. ``Tese (Doutorado em
%     Nutrição)'').
%   \item Ao utilizar o campo \texttt{illustrated}, deve-se preenchê-lo com
%     todo o texto desejado (e.g. ``il.'').
%   \item Além do campo \texttt{organization}, há o campo \texttt{nameaddon},
%     de modo que organizações como ``BRASIL. Supremo Tribunal de Justiça'' e
%     ``BRASIL. Supremo Tribunal Federal'' podem ter uma mesma organização,
%     diferenciando-se por esse campo. Isso permite que a primeira parte,
%     ``BRASIL'', não seja repetida várias vezes seguidas (e.g.
%     ``\underline{\hspace*{4em}}. Supremo Tribunal Federal'').
%   \item O \texttt{abntex2cite} às vezes usa o campo \texttt{type} como um
%     complemento da organização (um exemplo é a entrada
%     \texttt{brasil1988}).
%     Com o \texttt{biblatex-abnt} deve-se usar o campo \texttt{nameaddon} em
%     vez de \texttt{type}. Embora nesse caso (``Constituição (1988)'') o
%     campo
%     \texttt{type} faça mais sentido semanticamente, colocá-lo nessa
%     posição causaria problemas na organização alfabética da bibliografia,
%     já que o campo \texttt{type} às vezes aparece antes do título e às
%     vezes depois.
%   \item A recomendação do manual do \texttt{biblatex} (seção 2.3.3) é de
%     que, para autores corporativos, utilize-se os campos \texttt{author} e
%     \texttt{editor}, colocando o nome da organização entre chaves. Essa
%     opção tem a vantagem de permitir que se misture autores corporativos e
%     autores comuns (e.g. \texttt{editor = \{National Aeronautics and Space
%     Administration\} and John Doe}).
%   \item Quando o nome do autor, editor ou organização for muito grande para
%     usar nas citações, pode-se acrescentar os campos \texttt{short\-author}
%     e \texttt{short\-editor} (e.g. \texttt{author = \{National Aeronautics and
%     Space Ad\-min\-is\-tra\-tion\},} \texttt{short\-author = \{NASA\}}
%     imprimirá ``NASA'' nas citações e ``National Aeronautics and Space
%     Administration'' na bibliografia). O campo \texttt{org-short}, usado pelo
%     \texttt{abntex\-2\-cite}, é automaticamente convertido para
%     \texttt{shortauthor}.
%   \item O separador ``de'' faz com que tudo o que vem depois dele seja
%     considerado um único sobrenome (e.g. na entrada \texttt{alves1995} o nome
%     \texttt{Roque de Brito} \texttt{Alves} é impresso como ``BRITO ALVES,
%     Roque de''). Cf. \url{http://tex.stackexchange.com/q/308625/102699}.
%   \item Quando os campos \texttt{number}, \texttt{volume}, \texttt{chapter} e
%     \texttt{edition} contêm apenas números, eles são formatados
%     automaticamente (e.g. \texttt{edition = \{5\}} imprime ``5. ed.''). Quando
%     esses campos contêm letras, deve-se preencher todo o conteúdo desejado
%     (e.g. \verb"edition = {5th. ed}" imprime ``5th. ed.''). Os caracteres
%     \texttt{.,-/} podem ser usados e o campo ainda será considerado como
%     contendo apenas números.
%   \item Em entradas dos tipos \texttt{phdthesis}, \texttt{mastersthesis} e
%     \texttt{monography}, há a data de publicação, que aparece logo após o
%     título, e a data da defesa, que aparece por último. O
%     \texttt{abntex2cite}
%     às vezes usa o campo \texttt{year-presented} para diferenciar entre as
%     duas
%     datas e às vezes muda as opções do pacote para mostrar uma mesma data
%     em
%     uma ou outra posição. No \texttt{biblatex-abnt} pode-se usar os campos
%     de
%     datas usuais para a data de publicação, que aparece após o título, e
%     os campos \texttt{eventdate}, \texttt{eventmonth} e \texttt{eventyear}
%     para a
%     data da defesa, que aparece no final. O campo \texttt{year-presented},
%     usado pelo \texttt{abntex2cite}, é automaticamente convertido para
%     \texttt{eventyear}. (Cf. entradas \texttt{morgado1990},
%     \texttt{morgadob1990} e \texttt{morgadoc1990} no arquivo
%     \texttt{abnt-testcase.tex}.)
%   \item Em vez dos campos \texttt{reprinted-from} e \texttt{reprinted-text},
%     utilizados pelo \texttt{abntex2cite}, usa-se o campo \texttt{related} para
%     citar uma entrada relacionada e o campo \texttt{relatedtype} para
%     especificar a natureza dessa relação. O campo \texttt{reprinted-from} é
%     automaticamente convertido para \texttt{related}; esse campo deve conter a
%     chave da obra relacionada. O campo \texttt{relatedtype} pode conter
%     algumas
%     opções: \texttt{relatedtype=\{reprintfrom\}} imprime ``Separata de''
%     (esse texto também é usado como padrão quando se escreve qualquer coisa
%     no campo \texttt{reprinted-text}, usado no \texttt{abntex2cite});
%     \texttt{recensionof} imprime ``Recensão de''; \texttt{reviewof} imprime
%     ``Resenha de''; \texttt{reprintof} imprime ``Reimpressão de'';
%     \texttt{translationof} imprime ``Tradução de''. Outras possibilidades
%     podem ser encontradas no arquivo \texttt{brazilian.lbx}, incluso na
%     instalação padrão do \texttt{biblatex}.
%   \item No \texttt{biblatex}, as entradas \texttt{inbook} e
%     \texttt{incollection} não se comportam como no \texttt{bibtex}. Entradas
%     \texttt{inbook} também podem ter um \texttt{bookauthor}; elas estão para
%     as entradas \texttt{book} assim como as entradas \texttt{incollection}
%     estão para as entradas \texttt{collection} (Cf. seção 2.3.1 do manual
%     do \texttt{biblatex}). Para que o autor da obra principal seja
%     substituído pelos traços sublineares, é necessário que o campo
%     \texttt{bookauthor} seja igual ao campo \texttt{author}.
%   \item É possível usar o campo \texttt{furtherresp} como no
%     \texttt{abntex2cite}, mas é preferível usar os campos \texttt{editora},
%     \texttt{editoratype}, \texttt{editorb}, \texttt{editorbtype} etc. (e.g.
%     \verb"editora = {Ismael Cardim},  editoratype={coeditor}" imprimirá
%     ``Co-edição de Ismael Cardim'' na biblioqgrafia; cf. entradas
%     \texttt{hou\-a\-iss\-1996}, \texttt{koogan1998}, \texttt{ceravi1983},
%     \texttt{riofilme1998} e pesquisar pelo campo \texttt{editor\-a\-type} para
%     mais exemplos). Todas as opções para o campo \texttt{editor\-type} podem
%     ser encontradas no arquivo \texttt{abnt-brazilian.lbx}. Usar esses campos
%     em lugar do campo \texttt{furtherresp} assegurará que as entradas sejam
%     impressas de forma consistente, embora os exemplos da própria ABNT não o
%     sejam.
%   \item Para tradutores pode-se usar o campo \texttt{translator}.
%   \end{sloppypar}
% \end{itemize}
% % <<<2

\clearpage
\nocite{*}
\printbibliography

\end{document}% <<<1

% vim: set foldmarker=\ >>>,\ <<< foldmethod=marker :

